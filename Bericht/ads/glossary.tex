%!TEX root = ../dokumentation.tex

%
% vorher in Konsole folgendes aufrufen:
%	makeglossaries makeglossaries dokumentation.acn && makeglossaries dokumentation.glo
%
% oder makeglossaries dokumentation
%
% Um das auszuführen, braucht man ActivePerl oder ein ähnliches Programm. ActivePerl kann man bei MyIT Services beantragen

%
% Glossareintraege --> referenz, name, beschreibung
% Aufruf mit \gls{...}
%
% Glossar wird nur angezeigt, wenn Glossareintrag in Text referenziert wird
%
\newglossaryentry{Glossareintrag}{name={Glossareintrag},plural={Glossareinträge},description={Ein Glossar beschreibt verschiedenste Dinge in kurzen Worten}}

\newglossaryentry{ct}{name={Compiletime},description={Die Compiletime ist der Zeitpunkt, zu dem ein Programm vom Compiler in vom Computer ausführbaren Maschinencode übersetzt wird}}

\newglossaryentry{rt}{name={Runtime},description={Die Runtime ist der Zeitpunkt, zu dem ein bereits in Maschinencode übersetztes Programm vom Benutzer ausgeführt wird}}
